\documentclass{article}
\usepackage{ams thm}
\newtheorem{theorem} {Theorem} [section]
\newtheorem{corollary} {Corollary} [theorem]
\newtheorem{lemma} {theorem} [Lemma]
\newtheorem{definition} {Definition} [section]
\begin{document}
\section{Numbered theorems, definitions, corollaries and lemmas}
\begin{theorem} [Pythagorean theorem]
\label{pythagorean}
This is a theorem about right triangles and can be summarised in the next equation \[ x^2 + y^2 = z^2 \]
\end{theorem}
And a consequence of theorem \ref{pythagorean} is the statement in the next corollary.
\begin{corollary}
There's no right rectangle whose sides measure 3cm, 4cm, and 6cm.
\end{corollary}
\begin{lemma}
Given two line segments whose lengths are \(a\) and \(b\) respectively there is a real number \(r\) such that \(b=ra\).
\end{lemma}
\begin{definition} [prime no]
A prime no is a natural no greater than 1 that is not divisible by any no other than 1 itself.
\item example: 2,3,5 and 7 are prime nos
\end{definition}
\end{document}